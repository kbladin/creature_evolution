%______________________________________________________
%
%   Konstruerad av Marcus Bergner, bergner@cs.umu.se
%
%   Vid funderingar titta längst ned i denna fil,
%   eller skicka ett mail
%______________________________________________________
%

% lite inställningar
\documentclass[10pt, titlepage, oneside, a4paper]{article}
\usepackage[T1]{fontenc}
\usepackage[english,swedish]{babel}
\usepackage{amssymb, graphicx, fancyheadings}
\usepackage{graphicx}
\usepackage{listings}
\usepackage{float}
\usepackage[utf8]{inputenc}
\usepackage[T1]{fontenc}
\usepackage{graphicx}

\addtolength{\textheight}{20mm}
\addtolength{\voffset}{-5mm}
\renewcommand{\sectionmark}[1]{\markleft{#1}}

% \Section ger mindre spillutrymme, använd dem om du vill
\newcommand{\Section}[1]{\section{#1}\vspace{-8pt}}
\newcommand{\Subsection}[1]{\vspace{-4pt}\subsection{#1}\vspace{-8pt}}
\newcommand{\Subsubsection}[1]{\vspace{-4pt}\subsubsection{#1}\vspace{-8pt}}
	
% appendices, \appitem och \appsubitem är för bilagor
\newcounter{appendixpage}

\newenvironment{appendices}{
	\setcounter{appendixpage}{\arabic{page}}
	\stepcounter{appendixpage}
}{
}

\newcommand{\appitem}[2]{
	\stepcounter{section}
	\addtocontents{toc}{\protect\contentsline{section}{\numberline{\Alph{section}}#1}{\arabic{appendixpage}}}
	\addtocounter{appendixpage}{#2}
}

\newcommand{\appsubitem}[2]{
	\stepcounter{subsection}
	\addtocontents{toc}{\protect\contentsline{subsection}{\numberline{\Alph{section}.\arabic{subsection}}#1}{\arabic{appendixpage}}}
	\addtocounter{appendixpage}{#2}
}

% Ändra de rader som behöver ändras
\def\inst{teknik och naturvetenskap}
\def\typeofdoc{Projektuppsats}
\def\course{Projektrapport, TNM094}
%\def\title{Elements}
\def\name{\\ Namn, LiU-ID \\ Namn, LiU-ID \\ Creature Evolution}
\def\username{c00abc}
%\def\email{\username{}@cs.umu.se}
%\def\path{edu/KURS/lab1}
\def\graders{Karljohan Lundin Palmerius}


% Här börjar själva dokumentet
\begin{document}
	% skapar framsidan (om den inte duger: gör helt enkelt en egen)
	\begin{titlepage}
		\thispagestyle{empty}
		\begin{large}
			\begin{tabular}{@{}p{\textwidth}@{}}
				\textbf{LINKÖPINGS UNIVERSITET\hfill \today} \\
				\textbf{Institutionen för \inst} \\
				\textbf{\typeofdoc} \\
			\end{tabular}
		\end{large}
		\vspace{10mm}
		\begin{center}
			%\LARGE{\pretitle} \\
			\huge{\textbf{\course}}\\
			\vspace{10mm}
%			\LARGE{\title} \\
			\vspace{15mm}
			\begin{center}
				\begin{large}
					\begin{tabular}{ll}
						\textbf{} & \name \\
						%\textbf{E-mail} & \texttt{\email} \\
						%\textbf{Sökväg} & \texttt{\fullpath} \\
					\end{tabular}
				\end{large}
			\end{center}
			\vfill
			\large{\textbf{Handledare}}\\
			\mbox{\large{\graders}}
		\end{center}
	\end{titlepage}


	% fixar sidfot
	%\lfoot{\footnotesize{\name, \email}}
	\rfoot{\footnotesize{\today}}
	\lhead{\sc\footnotesize\title}
	\rhead{\nouppercase{\sc\footnotesize\leftmark}}
	\pagestyle{fancy}
	\renewcommand{\headrulewidth}{0.2pt}
	\renewcommand{\footrulewidth}{0.2pt}

	%Skapa sammanfattning
	\begin{abstract}
		Sammanfattning Sammanfattning Sammanfattning Sammanfattning Sammanfattning Sammanfattning Sammanfattning Sammanfattning Sammanfattning Sammanfattning Sammanfattning Sammanfattning Sammanfattning Sammanfattning Sammanfattning Sammanfattning Sammanfattning Sammanfattning Sammanfattning Sammanfattning Sammanfattning Sammanfattning Sammanfattning.
	\end{abstract}
	
	% skapar innehållsförteckning.
	% Tänk på att köra latex 2ggr för att uppdatera allt
	\pagenumbering{roman}
	\tableofcontents
	
	% och lägger in en sidbrytning
	\newpage

	\pagenumbering{arabic}

	% i Sverige har vi normalt inget indrag vid nytt stycke
	\setlength{\parindent}{0pt}
	% men däremot lite mellanrum
	\setlength{\parskip}{10pt}

	\Section{Inledning}		
	  	 
	  	 Inledning Inledning Inledning Inledning Inledning Inledning Inledning Inledning Inledning Inledning Inledning Inledning Inledning Inledning Inledning Inledning Inledning Inledning Inledning Inledning Inledning Inledning Inledning Inledning Inledning Inledning Inledning Inledning Inledning Inledning.
		
	\Section{Rubrik}
	
		Text. Referens \cite[sid. 122]{Ref1}

		\Subsection{Underrubrik}
				
			Text.

				\Subsubsection{UnderUnderrubrik}
				Text.
				
				\Subsection{Underrubrik}
				
				Text.
				%\ref{fig:figur1}. //Figurreferens
				
				%Figur-mall
				%\begin{figure}[H]
					%\centering
					%\includegraphics[scale=0.37]{Illustration.png}
					%\caption{\emph{Figurtext}}
					%\label{fig:figur1}
				%\end{figure}
		
	\newpage
	
	\begin{thebibliography}{9}
		\bibitem{Ref1} Exempelreferens.
	\end{thebibliography}
	
	\newpage
		\appendix
	
		\Section{Bilaga - Exempeltitel}

\end{document}

